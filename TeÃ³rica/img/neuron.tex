\begin{tikzpicture}[shorten >=1pt,->, node distance=\layersep]
    \tikzstyle{every pin edge}=[<-,shorten <=2pt]
    \tikzstyle{neuron}=[circle,draw,minimum size=25pt,inner sep=0pt]
    \tikzstyle{output neuron}=[draw,minimum size=25pt,inner sep=0pt]
    \tikzstyle{input neuron}=[neuron];
    \tikzstyle{hidden neuron}=[neuron];
    \tikzstyle{annot} = [text width=5em, text centered]

    % Draw the input layer nodes
    \node[input neuron] (I-1) at (0,-1) {$i_{1}$};
    \node[input neuron] (I-2) at (0,-2.5) {$i_{2}$};
    \node (I-etc) at (0,-3.5) {$\vdots$};
    \node[input neuron] (I-3) at (0,-4.7) {$i_{j}$};


    % Draw the hidden layer nodes
    \foreach \name / \y in {1,...,1}
        \path[yshift=0.cm]
            node[hidden neuron] (H-\name) at (\layersep,-2.5 cm) {$\sum$};

    % Draw the output layer node
    \node[output neuron,pin={[pin edge={->}]right:Saída}, right of=H-1] (O) {$\phi$};

    % Connect every node in the input layer with every node in the
    % hidden layer.
    \path (I-1) edge node[above]{$w_{1}$} (H-1);
    \path (I-2) edge node[above]{$w_{2}$} (H-1);
    \path (I-3) edge node[above]{$w_{j}$} (H-1);
    
    % Connect every node in the hidden layer with the output layer
    \foreach \source in {1,...,1}
        \path (H-\source) edge (O);

    % Annotate the layers
    \node[annot,above of=H-1, node distance=2.5cm] (hl) {Soma ponderada};
    \node[annot,left of=hl] {Entrada};
    \node[annot,right of=hl] {Função de activação};
\end{tikzpicture}